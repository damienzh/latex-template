\documentclass[10pt,a4paper,utf8]{article}
\usepackage[top=1in, bottom=0.7in, left=1.25in, right=1.25in,headheight=33pt]{geometry}
\usepackage[utf8]{inputenc}
\usepackage{CJKutf8} % chinese font support
\usepackage{appendix}
%=================figures subfigures pictures========================
\usepackage{graphicx}
%\usepackage{float}
\usepackage[font=small,floatrowsep=qquad,captionskip=5pt]{floatrow}
\usepackage{subcaption}
%====================================================================
\usepackage{datetime2} % for YYYY-MM-DD date display
%====================customise header footer=========================
%\usepackage{fancyhdr}
%\pagestyle{fancy}
%\fancyhead{}
%\fancyfoot{}
%\lhead{\includegraphics[height=1cm]{header/header}}
%\rfoot{\includegraphics[height=0.6cm]{header/footer_text}}
%\usepackage{lastpage}
%\lfoot{\thepage\ / \pageref{LastPage}}
%\renewcommand{\footrulewidth}{0.6pt}
%====================================================================
\usepackage{url}
\usepackage[unicode,pdftex]{hyperref}

\usepackage{cite}
\renewcommand{\contentsname}{目录}
\newcommand{\pic}{./pic/}

\begin{document}
\begin{CJK}{UTF8}{gbsn}
%=============================title page=============================
\begin{titlepage}
	\centering
	\includegraphics[scale=1]{./pic/white-block.png}\par
	\vspace{1cm}
	\begin{Huge}
		\textbf{中文编辑示例}\par
	\end{Huge}
	\vspace{15cm}
	\begin{LARGE}
		作者 \par\vspace{1cm}
		\today
	\end{LARGE}
\end{titlepage}
%====================================================================
\tableofcontents
%\listoftables
%\listoffigures
\thispagestyle{empty}
%====================================================================
\newpage
\setcounter{page}{1}
\section{前言}
主体是Latex的实例样本,中文支持只是辅助。

\begin{enumerate}
\item Tex版本:Tex Live 2018
\item 编辑软件:TexMaker
\item 操作系统:Ubuntu 16.04
\end{enumerate}
%====================================================================
\section{插图与表格}
\subsection{单独插图}
插入一张图片, 图 \ref{fig:bunny}。
\begin{figure}[H]
\centering
\begin{ffigbox}
{\caption{兔兔}\label{fig:bunny}}
{\includegraphics[width=0.5\linewidth]{\pic bunny.jpeg}}
\end{ffigbox}
\end{figure}
\subsection{并列插图}
并列插图,并且每张插图有各自的标题
\begin{figure}[H]
\CenterFloatBoxes
\begin{floatrow}
\begin{ffigbox}
{\caption{小猫}\label{fig:cat}}
{\includegraphics[width=\linewidth]{\pic cat.jpeg}}
\end{ffigbox}
\begin{ffigbox}
{\caption{小狗}\label{fig:dog}}
{\includegraphics[width=\linewidth]{\pic dog.jpeg}}
\end{ffigbox}
\end{floatrow}
\end{figure}
\subsection{并列插图与表格}
并排插入图片与表格,并且有各自的标题
\begin{figure}[H]
\CenterFloatBoxes
\begin{floatrow}
\begin{ffigbox}
{\caption{bunny}}
{\includegraphics[width=\linewidth]{\pic bunny}}
\end{ffigbox}
\begin{ttabbox}
{\caption{兔兔表格}}
{\begin{tabular}{|c|c|}
\hline 
\multicolumn{2}{|c|}{兔兔1} \\ 
\hline 
兔兔2 & 兔兔3 \\ 
\hline 
\end{tabular} }
\end{ttabbox}
\end{floatrow}
\end{figure}
%====================================================================
\section{引用}
使用BibTex文件\cite{citekey},注脚也可以使用链接\footnote{\url{http://bing.com}}
%====================================================================
\newpage
%\appendix

\bibliographystyle{IEEEtran}
\bibliography{bibfile}
%====================================================================
\end{CJK}
\end{document}
